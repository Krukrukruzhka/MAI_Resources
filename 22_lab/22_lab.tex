\documentclass[14pt, a4paper]{extreport}
\usepackage[a4paper, total={6in, 9in}]{geometry}
\usepackage[utf8]{inputenc}
\usepackage[russian]{babel}
\usepackage[T2A]{fontenc}
\usepackage{amssymb}
\usepackage{amsmath}
\usepackage{microtype}
\usepackage{fancyhdr}
\fancyhf{}

\thispagestyle{empty}
\usepackage{ragged2e}
\justifying

\begin{document}

\begin{flushleft}
    т.е. что произведение $gf$ абсолютно интегрируемо на отрезке 
    \newline [a, b].$\square$
\end{flushleft}

\newline Всё сказанное в этом пункте естественный образом переносится и на несобственные интегралы других видов, рассмотренных в п. 29.1, т.е. на интегралы вида (29.6), а также на интегралы общего типа (29.8). 
\newline
\\
\textbf{29.6 Исследование сходимости интегралов}
\newline
\\
Докажем один достаточный признак сходимости интегралов, называемый обычно \textit{признаком Дирихле.}
\\
\normalsize\textbf{Т\,Е\,О\,Р\,Е\,М\,А 5 (признак Дирихле).} Пусть:
\\
\textbf{1) }\textit{функция f непрерывна и имеет ограниченную первообразную F при x $\ge$ a} \textbf{;}
\\
\textbf{2) }\textit{функция g непрерывно дифференцируема и убывает при x $\ge$ a} \textbf{;}
\\
\begin{left}
  \textbf{3) } $$\lim\limits_{x\to +{\infty}} g(x)=0.$$
\end{left}
\newline

\\\textit{Тогда сходится интеграл}
\newline
\begin{equation*}
    \int\limits_{a}^{+{\infty}} f(x)g(x) \,dx. \tag{\textbf{29.34}}
\end{equation*}

\begin{flushleft}
    {\largeД\,о\,к\,а\,з\,а\,т\,е\,л\,ь\,с\,т\,в\,о. }Прежде всего заметим, что, в силу сделанных предположений, функция $fg$ непрерывна, а значит, и интегрируема по Риману на любом отрезке [a,\,b], a < b < {+\infty}, и поэтому имеет смысл говорить о несобственном интеграле (29.34).
\end{flushleft}
Проинтегрировав по частм произведение $f(x)g(x)$ на отрезке [a,\,b], получим
\begin{equation*}
    \int\limits_a^b\,f(x)g(x)\,dx=\int\limits_{a}^{b}\,g(x)\,dF(x) |\limits_{a}^{b}\,-\,\int\limits_{a}^{b}\,F(x)g'(x)\,dx. \tag{\textbf{29.35}}
\end{equation*}
    
    \,\,\,Исследуем поведение обоих слагаемых правой части при {b\to +{\infty}}. В силу ограниченности функции F (см. условие 1 теоремы), 
\begin{equation*}
    M = sup |F(x)| < {+\infty}.
\end{equation*}
Из условий 2 и 3 теоремы следует, что функция $g$ не отрицательна для всех $x$$\ge$$a$, в частности $g(b) \ge$0;\,поэтому
\begin{equation*}
    |g(b)F(b)| \ge Mg(b).
\end{equation*}
\begin{center}
    \line(1, 0){100} \\
    \textit{671}
\end{center}

\end{document}